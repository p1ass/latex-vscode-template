\documentclass[sotsuron]{kuee}

\title{\LaTeX を用いた修論$\cdot$卒論の執筆}
\etitle{Usage of The \LaTeX{} Style File for KUEE}
\author{岸 直輝}
\eauthor{Jiro Denki}
\professor{電気 太郎 教授}
% \course{京都大学大学院情報学研究科}
% \department{知能情報学専攻}
\date{令和2年4月14日}
 

%%% 本文
\begin{document}
\maketitle			% 表題を出力
\begin{eabstract}		% 英文要旨を出力
This document briefly explains the usage of the \LaTeX{} style file
for KUEE bacheler thesis and master thesis.
\end{eabstract}
\tableofcontents		% 目次を出力

\chapter{はじめに}
\label{chap:intro}

この文書は,京都大学電気電子工学科の修論$\cdot$卒論作成用 \LaTeX2e{}
クラスファイルの利用方法についての説明書です.
%
修論$\cdot$卒論の正確なフォーマットの指定は,事務室から配布される手引
を参照してください.


\chapter{修論$\cdot$卒論クラスファイルの利用方法}
\label{chap:usage}

\section{インストール}

配布キットには,表~\ref{tab:kit} のファイルが含まれています.
\begin{table}
  \caption{配布キットのファイル一覧}\label{tab:kit}
  \begin{center}
    \begin{tabular}{ll}
      \verb+kuee.cls+ & 修論$\cdot$卒論用 \LaTeX2e{} クラスファイル \\
      \verb+kueethesis.bst+ & 修論$\cdot$卒論用 \JBibTeX{} 文献スタイルファイル \\
      \verb+sample.tex+ & 使用の手引(このドキュメント)を作るファイル \\
      \verb+sample.bib+ & 使用の手引の参考文献を収めたファイル \\
    \end{tabular}
  \end{center}
\end{table}

このクラスファイルを使用するため,{\ttfamily kuee.cls} と {\ttfamily
kueethesis.bst} を,環境変数 {\ttfamily TEXINPUTS} で指定されたディレ
クトリ,または修論$\cdot$卒論の原稿と同じディレクトリにコピーして下さい.

配布キットの文字コードは Unicode になっています.利用環境に応じて,適
切に文字コードを変換してください.


\section{ドキュメントスタイル}

ドキュメントスタイルは,オプションとして指定します.修論の場合は
{\ttfamily shuuron},卒論の場合は {\ttfamily sotsuron} を用います.例
えば,卒論の場合は次のように指定して下さい.
\begin{quote}
\begin{verbatim}
\documentclass[sotsuron]{kuee}
\end{verbatim}
\end{quote}
指定を省略すると,修論用のスタイルが選択されます.


\section{文字数,行数の設定}

1行の文字数と1ページの行数を指定する場合は,\verb+\begin{document}+ よ
り先に次のように指定します.これは,デフォルトと同じ1行36文字,1ページ
32行の設定例です.
\begin{quote}
\begin{verbatim}
\charsinline{36}
\linesinpage{32}
\end{verbatim}
\end{quote}
ただし,\TeX のページ分割のためにすべてのページが必ずしも設定通りの行
数にはなりません.ASCII \JTeX では句読点のカーニングの伸縮のため,すべ
ての行が必ずしも設定通りの文字数にはなりません.なお,1行の文字数はNTT
\JTeX では最大38文字,ASCII \JTeX では最大37文字で,これを越えると隣合
う文字同士が重なってしまいます.


\section{表紙}
表紙は \verb+\maketitle+ コマンドによって出力されます\footnote{表紙ペー
ジのページ番号は0ですが,出力されません.}.

表紙を出力するコマンド \verb+\maketitle+ よりも先に,タイトル(日本語お
よび英語),著者氏名(日本語および英語)などを次のように指定する必要があ
ります.
\begin{quote}
\begin{verbatim}
\title{\LaTeX を用いた修論$\cdot$卒論の執筆}
\etitle{Usage of The \LaTeX{} Style File for KUEE}
\author{電気 次郎}
\eauthor{Jiro Denki}
\professor{電気 太郎 教授}
\date{平成13年12月18日}
\end{verbatim}
\end{quote}
なお,英語タイトルおよび英語著者氏名は,表紙ではなく,英文要旨を出力す
る時に用いられます.

上記の例のように,研究科(学部)と専攻(学科)の指定を省略すると,修論作成
時には,工学研究科~電気工学専攻が指定されたと見なされます.卒論作成時に
は,工学部~電気電子工学科が指定されたと見なされます.
%
研究科(学部)を指定する場合は \verb+\course+ コマンドを,専攻(学科)を指
定する場合は \verb+\department+ コマンドを,以下のように使用してくださ
い.
\begin{quote}
\begin{verbatim}
\course{京都大学大学院情報学研究科}
\department{知能情報学専攻}
\end{verbatim}
\end{quote}


\section{英文要旨}
英文要旨は,\verb+eabstract+ 環境を用いて記述します.


\section{目次}
目次は \verb+\tableofcontents+ コマンドによって出力されます\footnote
{目次ページのページ番号は,ローマ数字で出力されます.}.謝辞,参考文献,
付録なども目次に掲載されます.


\section{本文}
本文は通常の\LaTeX のテキストとして記述します\footnote{本文ページのペー
ジ番号は1から始まり,アラビア数字で出力されます.}.
いくつかの点でj-report/jreport/reportスタイルとの違いがあります.

\subsection{章題}
章題は次のように出力されます.
% (英文用では章番号の部分が{\LARGE\bfseries Chapter~3}のようになります).

\chapterhead{3}{修論・卒論クラスにおける章題}

章題が1行に収まり切らない場合には次のように改行されて出力されます.

\chapterhead{3}{修論・卒論クラスにおける1行に収まり切らないような長い章題}

\subsection{脚注}

脚注は章ごとにカウントされ,マークは$^{*}$, $^{**}$, $^{***}$,
$^{\dagger}$, $^{\dagger\dagger}$, $^{\dagger\dagger\dagger}$,
$^{\ddagger}$, \ldots のようになります.

\subsection{図表}

通常の \LaTeX を利用する場合と同様,本文中の適当な場所に記述して下さい.
全ての図表は,\TeX によって自動的に論文の末尾に移動されます.例えば,
図~\ref{fig:example} は,この段落の直後で定義されていますが,実際の整
形結果では論文末尾に移動しているはずです.

\begin{figure}
  \begin{center}
    \unitlength=1mm
    \begin{picture}(100,100)(-50,-50)
      \put(0,-50){\vector(0,1){100}}
      \put(-50,0){\vector(1,0){100}}
      \put(0,0){\makebox(0,0)[rt]{$o$}}
      \put(50,0){\makebox(0,0)[lt]{$x$}}
      \put(0,50){\makebox(0,0)[rb]{$y$}}
      \put(0,0){\vector(2,1){30}}
      \put(30,15){\makebox(0,0)[lt]{$\vec{a}$}}
      \put(0,0){\vector(1,2){15}}
      \put(15,30){\makebox(0,0)[lt]{$\vec{b}$}}
      \thicklines
      \put(0,0){\vector(1,1){45}}
      \put(45,45){\makebox(0,0)[lt]{$\vec{a}+\vec{b}$}}
    \end{picture}
  \end{center}
  \caption{figure 環境の例}
  \label{fig:example}
\end{figure} % この図は最後に出力される

大量の図表を張り付けると,以下のようなエラーが発生することがあります.
\begin{quote}
\begin{verbatim}
! LaTeX Error: Too many unprocessed floats.
\end{verbatim}
\end{quote}
\LaTeX が図表を組み版する時は,前後の文章の量を見ながらオプションで指
定された条件に合う場所が出てくるまでメモリーに図表を保存しています。上
記エラーは,図表が数ページにわたって連続して現われ,メモリーが不足する
と発生します.このエラーが発生した時は,適当な位置に 
\verb+\clearfigurepage+ コマンドを挿入してください.このコマンドは,図
表ページを指定された個所で強制的に分割し,組版処理を行うように指示しま
す.

% なお,この図表の定義場所に関する点が,\LaTeX{}209 用に配布されていた従
% 来の {\ttfamily kueethesis.sty} から最も大きく異なっています.従来版で
% は,\verb+\figureandtable+ と \verb+\figureandtableof+ の2つのコマンド
% を利用して,手作業で論文末尾に図表を記述する必要がありました.互換性を
% 保つため,これらのコマンドは一応定義されていますが,出力に矛盾を生じる
% 可能性がありますので,決して使用しないで下さい.研究室の先輩の原稿など
% を参考にして作業している場合は,特に注意して下さい.


\section{謝辞}
謝辞は,\verb+acknowledgements+ 環境を用いて記述します.

\section{相互参照}
\label{cross_reference}
相互参照は,通常の\LaTeX{}文書と同様に,\verb+\label+ コマンドと
\verb+\ref+ コマンドを用いて行います.例えば,章番号を参照する場合には,
以下のように \verb+\chapter+ コマンドの直後に \verb+\label+ コマンドを
配置してラベルを宣言します.
\begin{quote}
\begin{verbatim}
\chapter{はじめに}
\label{chap:intro}
\end{verbatim}
\end{quote}
その上で,参照したい箇所に,\verb+\ref+ コマンドを以下のように配置しま
す.
\begin{quote}
\begin{verbatim}
\ref{chap:intro}章では,本利用説明書の位置づけについて述べています.
\end{verbatim}
\end{quote}
\verb+\ref{chap:intro}+ は実際の章番号に置換されて,以下のように組版されます.
\begin{quote}
\ref{chap:intro}章では,本利用説明書の位置づけについて述べています.
\end{quote}
詳しくは,本利用説明書内の利用例および,
\LaTeX2e{}美文書作成入門\cite{GuideBook}の第10章などを参照してください.


\section{参考文献}
参考文献は \verb+thebibliography+ 環境を用いて直接記述するか,
\JBibTeX{} システムを用いて作成することができます\footnote{(J)\BibTeX 
の使い方については \LaTeX ブック\cite{LaTeX}の付録Bなどを参照.}.参考
文献のラベルは `[1]' のように出力されます.引用も同様に
`[1]'のように出力されます.

2017年以前の旧版クラスファイルでは,文献のラベルは`1)',引用は`$^{\mbox{\scriptsize 1)}}$'
のようになっていました.
旧版クラスファイルと同じ出力を得るためには,ドキュメントクラスのオプションに
以下のように{\ttfamily oldcite}を指定してください.

\begin{quote}
\begin{verbatim}
\documentclass[oldcite]{kuee}
\end{verbatim}
\end{quote}

\section{付録}

付録は \verb+\appendix+ コマンドの後に記述します\footnote{付録ページの
ページ番号は本文から継続し,アラビア数字で出力されます.}.付録の各項
目は \verb+\chapter+ コマンドによって分割して記述します.付録がただ1項
目からなる場合にも項目の始めに \verb+\chapter+ コマンドを用いて項目名
を指定して下さい.


\chapter{おわりに}
\label{chap:conclusion}

\LaTeX{}209用スタイルファイルの利用説明書\cite{OldTebiki}には,次のよ
うに書かれていました.
\begin{quote}
  京大電気系学科の修論$\cdot$卒論に\LaTeX が使われ出して4年目になりま
  す.最初のころはPC98上のアスキー日本語Micro-\TeX を使ってちんたらやっ
  ていたものですが,最近ではUnixマシンおよびUnix上のNTT \JTeX, ASCII
  \JTeX が広く普及し,修論・卒論を\LaTeX で書こうという人はかなり多く
  なっているものと思います.
\end{quote}
すなわち,30年の長きにわたって,\LaTeX{}が修論$\cdot$卒論の作成に用い
られていることになります.これは,\LaTeX{}の論理マークアップという考え
方が,論文作成と親和性が高いことの証明であると,著者は考えます.例えば,
論文作成時には,論旨を明確化するために,章や節を単位として順序を頻繁に
変更する必要が生じます.相互参照(\ref{cross_reference}節)を正しく活用
していれば,どのように順序を変更しても常に正しく番号が付番されるので,
納得がいくまで順序を考えることができます.

このクラスファイルを活用し,皆さんがより良い修論$\cdot$卒論を執筆され
ることを願っています.



%======================================================================
%		謝辞
%======================================================================
\begin{acknowledgements}
  オリジナルの \LaTeX{}209 用スタイルファイル {\ttfamily
  kueethesis.sty} の配布キットを作成された,傳康晴さんに感謝します.

  図表を論文の末尾に移動する方法について, {\ttfamily endfloat.sty} を
  参考にさせて頂きました.筆者の James Darrell McCauley さんに感謝しま
  す.また,改造方法について \TeX{} FAQ 掲示板でアドバイスをくださった 
  misc さんに感謝します.

  このクラスファイルの初期の版から現在の版に至るまで,各年度の長尾研究
  室(現$\cdot$言語メディア研究室)をはじめ多くの研究室の人たちから,数々
  の貴重なコメントを頂きました.関係者各位に感謝します.
\end{acknowledgements}



%======================================================================
%		参考文献
%======================================================================
\bibliographystyle{kueethesis}
\bibliography{sample}



%======================================================================
%		付録
%======================================================================
\appendix
\chapter{改版履歴}\label{chap:history}
\begin{description}
  \item[1991年] 傳康晴\footnote{工学研究科\ 電子工学専攻 長尾研(当時).
	     現在,千葉大学.}が,\LaTeX{}209 用修論$\cdot$卒論スタイ
	     ルファイル {\ttfamily kueethesis.sty} と文献用スタイルファ
	     イル {\ttfamily kueethesis.bst} を作成\cite{OldTebiki}.
  \item[2001年] 土屋雅稔\footnote{情報学研究科 知能情報学専攻 言語メディ
	     ア研(当時).現在,豊橋技術科学大学.{\ttfamily
	     tsuchiya@tut.jp}}が,\LaTeX2e{} 用修論$\cdot$卒論クラスファ
	     イル{\ttfamily kuee.cls} を作成\cite{Tebiki2004}.
	     \newcounter{tsuaffil}
	     \setcounter{tsuaffil}{\value{footnote}}
  \item[2017年] 土谷亮\footnote{情報学研究科 通信情報システム専攻 小野
	     寺研.現在,滋賀県立大学.{\ttfamily
	     tsuchiya.a@e.usp.ac.jp}}が,参考文献の形式を改良.
  \item[2018年] 土屋雅稔\footnotemark[\value{tsuaffil}]{}が,英文要旨
	     に対応するためのコマンドを追加.
\end{description}


\chapter{レイアウト・パラメータ}\label{chap:layout}
デフォルトで利用される本文ページ,図・表ページのレイアウト・パラメータ
はそれぞれ表~\ref{tab:text},\ref{tab:fig} のようになっています.

\begin{table}
  \caption{本文ページのデフォルト・レイアウト}\label{tab:text}
  \begin{center}
    \begin{tabular}{|l|r|}
      \hline
      \verb+\textwidth+ & 424pt \\ \hline
      \verb+\textheight+ & 604pt \\ \hline
      \verb+\oddsidemargin+ & 0.5cm \\ \hline
      \verb+\evensidemargin+ & 0.5cm \\ \hline
      \verb+\topmargin+ & 0pt \\ \hline
      \verb+\headheight+ &12pt \\ \hline
      \verb+\headsep+ & 25pt \\ \hline
      \verb+\footskip+ & 30pt \\ \hline
    \end{tabular}
  \end{center}
\end{table}

\begin{table}
  \caption{図・表ページのデフォルト・レイアウト}\label{tab:fig}
  \begin{center}
    \begin{tabular}{|l|r|}
      \hline
      \verb+\textwidth+ & 424pt + 1cm \\ \hline
      \verb+\textheight+ & 604pt + 67pt \\ \hline
      \verb+\oddsidemargin+ & 0pt \\ \hline
      \verb+\evensidemargin+ & 0pt \\ \hline
      \verb+\topmargin+ & 0pt \\ \hline
      \verb+\headheight+ & 0pt \\ \hline
      \verb+\headsep+ & 0pt \\ \hline
      \verb+\footskip+ & 0pt \\ \hline
    \end{tabular}
  \end{center}
\end{table}

\end{document}
% Local Variables:
% fill-column: 70
% End:
